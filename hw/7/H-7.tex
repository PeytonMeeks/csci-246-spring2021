\documentclass{article}
\usepackage{../fasy-hw}

%% UPDATE these variables:
\renewcommand{\hwnum}{7}
\title{Discrete Structures, Homework \hwnum}
\author{Peyton Meeks}
\collab{n/a}
\date{due: 16 April 2021}

\begin{document}

\maketitle

This homework assignment should be
submitted as a single PDF file both to D2L and to Gradescope.

General homework expectations:
\begin{itemize}
    \item Homework should be typeset using LaTex.
    \item Answers should be in complete sentences and proofread.
    \item You will not plagiarize, nor will you share your written solutions
        with classmates.
    \item List collaborators at the start of each question using the \texttt{collab} command.
    \item Put your answers where the \texttt{todo} command currently is (and
        remove the \texttt{todo}, but not the word \texttt{Answer}).
\end{itemize}


% ============================================
% ============================================
\collab{n/a} \nextprob{Graphs}
% ============================================
% ============================================

Often, in order to transform real-world problems into ones that can be analyzed
on computers, you need to design a representation of the data that helps
illuminate the patterns.  One common representation is a graph.  Suppose you own
a movie store.  You have records of every movie purchased, how much it was
purchased for, and who purchased it.  Describe two different graphs that you can
create to represent this data.  Please make sure that the nodes in the two
graphs represent different things.

\paragraph{Answer}

The first graph could be a scatterpot with the x-axis representing each movie, and the y-axis representing the amount the movie was purchased for. Using a scatterplot allows each movie to have multiple nodes seperated by the selling point of each movie if there is a variance. The second graph would be another scatterplot relating each movie to the person who purchased it. This allows each node to be seperated by the person who purchaced the movie, and related by each movie title sold.


% ============================================
% ============================================
\collab{n/a} \nextprob{Equivalence Class}
% ============================================
% ============================================

Define a relation $R$ between all simple graphs where two graphs $g$ and $h$ are
related (denoted $gRh$) if and only if $g$ and $h$ have the same number of
connected components.

\begin{enumerate}

    \item Prove that this is an equivalence relation.

        \paragraph{Answer}

        Suppose not, that is suppose that there exits a relation $R$ on graphs $g$ and $h$ such that $g$ and $h$ do not have the same number of connected components. Then the graphs must be symetric, meaing for all components in $g$ and $h$, $g\leftarrow h$ if and only if $h \leftarrow g$ which is not true since $g$ and $h$ do not have the same number of connected components. Therefore by prove by contradiction, $gRh$ if and only if $g$ and $h$ have the same number of connected components. 

    \item Describe a scenario where you might use this equivalence relation.

        \paragraph{Answer}

        One scenario would be if you were comparing the number of sales of a certain amount of products from year to year.

\end{enumerate}
% ============================================
% ============================================
\collab{n/a} \nextprob{Pseudocode}
% ============================================
% ============================================

Recall the binary search algorithm.

\begin{enumerate}
    \item Using the algorithm/algorithmic environment,
        give pseudocode using a for loop.

        \paragraph{Answer} My algorithm for binary search using a for loop is given in \algref{forloop}.

         \begin{algorithm}
            \caption{\textsc{BinarySearchFor}$(A)$}\label{alg:forloop}
            \begin{algorithmic}
				\State \todo{Answer here}
			 \end{algorithmic}
        \end{algorithm}
    \item Using the algorithm/algorithmic environment, give pseudocode using a while loop.

        \paragraph{Answer} My algorithm for binary search using a while loop is given in \algref{whileloop}.

        \begin{algorithm}
            \caption{\textsc{BinarySearchFor}$(A)$}\label{alg:whileloop}
            \begin{algorithmic}
                  \State int $start=0$,$end=array.length-1$
				\While{$start <= end$}
					\State int $middle = start + (end - start)/2$
					\If {$array[middle] == A$}
						\State \Return $true$
					\ElsIf{$array[middle] > A$}
						\State $end = middle -1$
					\Else 
						\State {$start = middle +1$}
					\EndIf
				\EndWhile \State
				\Return $false$
            \end{algorithmic}
        \end{algorithm}
		\footnote{“Binary Search Using a for Loop.” Binary Search Using a for Loop (Beginning Java Forum at Coderanch), \url{coderanch.com/t/608725/java/Binary-Search-loop}.}

    \item Using the algorithm/algorithmic environment, give pseudocode using
        recursionn.

		\paragraph{Answer} My algorithm for binary search using recursion is given in \algref{recursion}.

        \begin{algorithm}
            \caption{\textsc{BinarySearchFor}$(A)$}\label{alg:recursion}
            \begin{algorithmic}
				\Require $start = 0, end = array.length -1$  
                  \State int RecursiveBinarySearch(int $array[]$, int $start$, int $end$, $A$) 
					\State int $middle = start + (end - start)/2$
					\If{$array[middle] == A$}
						\State \Return $true$
					\ElsIf{$A < array[middle]$}
						\State \Return $RecursiveBinarySearch(array, start, middle - 1, A)$
					\Else
						\State \Return RecursiveBinarySearch(array, mid + 1, end, A)
					\EndIf
					\State \Return $false$
            \end{algorithmic}
        \end{algorithm}
		\footnote{Arora, Nishtha. “Iterative and Recursive Binary Search Algorithm.” OpenGenus IQ: Learn Computer Science, OpenGenus IQ: Learn Computer Science, 26 Apr. 2020, \url{iq.opengenus.org/binary-search-iterative-recursive/}.} 

    \item What is the loop invariant of your second algorithm? (Proofs are not
        necessary, just stating the LI is required here.  As usual, for partial
        credit of an incorrect answer, reasoning will need to be justified).

        \paragraph{Answer}

        The loop invariant of the second algorithm is $start \leq end$. Resources used to construct the algorithms are in the footnote. 
\end{enumerate}

% ============================================
% ============================================
\collab{n/a} \nextprob{Four Colors Suffice}
% ============================================
% ============================================

Read Chapter $11$ of \emph{Four Colors Suffice}.

\begin{enumerate}

    \item What is a proof?

        \paragraph{Answer}

        A proof is evidence or an argument of evidence trying to establish the truthfulness of a statement.


    \item Choose one concept that was described in both FCS and in Epp.
        Compare and contrast their explanations of the concept.

        \paragraph{Answer}

        \todo{your answer here}

\end{enumerate}

% ============================================
% ============================================
\collab{n/a}
\nextprob{Thomas Bayes}
% ============================================
% ============================================

Write a short (1-2 paragraph) biography of Thomas Bayes.
\textbf{In your own words}, describe who they are and why they are important in
the history of computer science.

If you use external resources, please provide
proper citations. If you do not use external sources, please write ``I did not
use any sources to write this biography'' as the last sentence of the
biography.

\paragraph{Answer}

Thomas Bayes was a philosopher and statistician in the early 1700's. Born in Hertfordshire, England in 1701, Bayes would then enroll at the University of Edinburgh in 1719. While at the university, Bayes studied logic and theology until returning home to help his father who was a Presbyterian minister. In 1734 Bayes moved to Tunbrige Wells where he became a minister as well. Bayes took a great interest in mathematics, specifically probability. Many are unsure what led him to do so, however Bayes went on to publish \textsc{An Introduction to the Doctrine of Fluxions, and a Defence of the Mathematicians Against the Objections of the Author of The Analyst} which helped defend the foundation of Sir Issac Newton's calculus. His most famous findings, for which are named after him, were not published until after his death in 1761. Bayes' Therom is an extremly important tool to both mathematicians and computer scientists alike due to the fact that it can be used to find the probability of certain events or even relating an event back to the original problem. This allows us to have a uniform way of finding solutions to these problems and describing them justly. \footnote{“Thomas Bayes.” Wikipedia, Wikimedia Foundation, 29 Mar. 2021, \url{en.wikipedia.org/wiki/Thomas_Bayes}.}

% %% ... the bibliography
% \newpage
% \bibliographystyle{acm}
% \bibliography{biblio}

\end{document}

