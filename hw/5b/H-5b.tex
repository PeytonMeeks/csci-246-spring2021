\documentclass{article}
\usepackage{../fasy-hw}

%% UPDATE these variables:
\renewcommand{\hwnum}{5b}
\title{Discrete Structures, Homework \hwnum}
\author{Peyton Meeks}
\collab{n/a}
\date{due: 29 March 2021 (Monday)}

\begin{document}

\maketitle

This homework assignment should be
submitted as a single PDF file both to D2L and to Gradescope.

General homework expectations:
\begin{itemize}
    \item Homework should be typeset using LaTex.
    \item Answers should be in complete sentences and proofread.
    \item You will not plagiarize.
    \item List collaborators at the start of each question using the \texttt{collab} command.
    \item Put your answers where the \texttt{todo} command currently is (and
        remove the \texttt{todo}, but not the word \texttt{Answer}).
\end{itemize}

{\color{blue} This homework is a resubmission of previously assigned homework
questions.  We encourage you to use the previous feedback to improve your
solutions to these problems, as well as to discuss the solutions with each other.
As usual, don't hesitate to ask questions! And, the write-up that you submit
MUST be in your own words.  In
this homework, there are eleven proofs (some are grouped together as one
question).  Each question will be graded using the following scheme:
\begin{itemize}
    \item No credit (+0pts). Either no solution or a major logical error was found
        (e.g., started with what needed to be proven).
    \item Low Pass (+6pts). Some of proof present, but either skips a step (or
        two) or
        has multiple small errors.
    \item Pass (+8pts). Mostly correct. May have small errors.
    \item High Pass (+10pts). Proof is exemplary.
\end{itemize}
}


% ============================================
% ============================================
\nextprob{(1-3) A Proof}
\collab{n/a}
% ============================================
% ============================================

Prove that $6\Z \subset 2\Z$.

\paragraph{Answer}

Suppose that $6\Z \subset 2\Z$ and let $n \in 6\Z$. By definition of a set, n must be some integer $x$ multiplied by 6 to be in $6\Z$. By definition of a set we know that for an element to exist in $2\Z$ it must be some integer $y$ multiplied by 2. Therefore for $6\Z \subset 2\Z$ to be true by definition of a subset, $6x \in 2\Z$. By substitution, $6x=2y$. Using algerbra, $y=3x$. Since $x$ and $y$ are any integer and by definition of multiplcation of integers, we can rewrite the equation as $y=y$. Therefore $6\Z subset 2\Z$ is true.

% ============================================
% ============================================
\collab{n/a}
\nextprob{(2-2) Existential Statements}
% ============================================
% ============================================

Are the following statements true or not true?    Prove or disprove.

\begin{enumerate}

    \item All even integers are equal to an odd integer plus one.

        \paragraph{Answer}
       For every integer $n$, if $n$ is odd then $n+1$ is an even integer. By definition of even, we know that an even integer is some integer $k*2$. By definition of odd, we know that an odd integer $n=2s+1$ for some integer $s$. Then by substitution $(2s+1)+1=2k$. Using algerbra to simplify we have $2(s+1)=2k$. Therefore $k=s+1$ for some integers $k$ and $s$ and by substitution we have $2k=2k$. Therefore for any odd integer $n$, $n+1$ is an even integer.

    \item All horses are the same color.

        \paragraph{Answer}
        If one horse is brown, then all horses are brown. This horse is brown, the other horse is
black. Therefore not all horses are the same color. Modus Tonens


\end{enumerate}

% ============================================
% ============================================
\collab{n/a} \nextprob{Definitions}
% ============================================
% ============================================
Use the definitions provided in the course textbook to prove that every prime
number except~$2$ is odd.

\paragraph{Answer}

Say there exists an even prime integer $q>2$, then by the definition of an even number, for all intergers $k$, $q$ is even if $q=2k$. By the definition of prime numbers: for some interger $q$, if $q>1$ and for all other positve intergers $(r,s) q=rs$ then $r$ or $s$ equals $q$ and $1$. By substitution, $2k=rs$. Therfore $r,s$ have to equal 2 and $k$ and from the definition of a prime number we know that $r,s$ can only equal 2 and $q$. So by proof of contradiction, the only even prime number is 2. Therefore all prime numbers except for 2 are odd.

% ============================================
% ============================================
\collab{n/a}
\nextprob{(3-3) Four Colors Suffice}
% ============================================
% ============================================
Read Chapters $2$ and $3$ of \emph{Four Colors Suffice} and answer the following questions:

\begin{enumerate}
    \item[4.] Prove or disprove: all plane graphs (maps) are three-colorable.

        \paragraph{Answer}
        Suppose not, that is suppose there exists a plane graph that is not three colorable. By the four color therom, we know that any plane graph is at most four colorable therefore is goes to stand that there is a plane graph that is four colorable. Therefore if a graph requires four colors, it cannot be colored by only three. Since the supposition is true the proposition must be false and all plane graphs are not three colorable.

    \item[5.] Assuming the four color theorem holds, prove or disprove: six colors
        suffice to color a plane graph.

        \paragraph{Answer}
        Since four colors suffice for all plane graphs then six colors would more than suffice to the
point of redundancy.


    \item[7.] Euler's formula states that if we have a map on the sphere or plane
        and count the exterior face as a face, then F-E+V=2.  Does this equation
        hold if the map is drawn on a M\"obius band? Why or why not? (Note:
        here, the boundary of the M\"obius band must be represented in the graph
        defining the map, and no ``country'' can be on the same side of a single
        edge.)

        \paragraph{Answer}
       No it does not, becuase when you create the M\"obius band Euler's Formula is $2-1+0$, which does not equal $2$ since the band only has one face and one edge within it and no verticies.


\end{enumerate}




% ============================================
% ============================================
\collab{n/a} \nextprob{(4-2) Max of a Subset}
% ============================================
% ============================================

Let $(B,\leq)$ be a totally ordered finite set. Prove the following
statement: For all nonempty subsets $A \subseteq B$, the following inequality
holds: $\max(A) \leq \max(B)$.

\paragraph{Answer}

Suppose there exists a subset $C \subseteq B$ for which $\max(C) > \max(B)$. If $B$ is truly a finite set then by definition of a finite set, $\max(B) \in B$. Since $C \subseteq B$, $\max(C)$ must exist within $B$. Therefore $\max(C)$ cannot be $> \max(B)$ and by proof by contradiction, for any subset $A \subseteq B$, $\max(A)$ must be $\leq \max(B)$.

% ============================================
% ============================================
\collab{n/a} \nextprob{(4-3) Fibonacci}
% ============================================
% ============================================

The Fibonacci numbers are defined as follows:
$$
    F_i = \begin{cases}
            1 & i \in \{1,2\} \\
            F_{i-1}+F_{i-2} & \text{otherwise}
          \end{cases}
$$

Prove $\sum_{i=1}^n F_i = F_{n+2}-1$.

\paragraph{Answer}

Let $n$ be an integer $\geq 1$ and assume that $\sum_{i=0}^n F_i = F_{n+2}-1$ is true. By the definiton of Fibonnaci numbers: $F_n =F_{n-1}+F_{n-2}$ for all integers $n>2$. Consider $F_{n+2}$, by the recursive formula $F_{n+2}=F_{n+1}+F_n$ and $F_{n+1} =F_n+F_{n-1}$. By using algerbra $F_{n+2}=F_n+F_{n-1}+F_n$. By inductive assumption, $\sum_{i=0}^n F_i = F_n+F_{n-1}+F_n-1$. Let $n=1$, such that $\sum_{i=0}^1 F_i = F_1+F_{1-1}+F_1-1$.Calulating the left side of the equation,  $\sum_{i=0}^n F_i =0+1=1$. Calculating the right side of the same equation,  $F_1+F_{1-1}+F_1-1= 1+0+1-1=1$. Both sides of the equation equal the same thus  $\sum_{i=0}^n F_i = F_{n+2}-1$ is true.


% ============================================
% ============================================
\collab{n/a} \nextprob{(4-4) US Coins}
% ============================================
% ============================================

Consider the four smallest denominations of US coins: $D=\{1,5,10,25\}$.  Prove, using
induction, that, for each $n \geq 1$, you can make $n$ cents using at most four
pennies.

\paragraph{Answer}

Suppose that if $n \geq 1$ you can make $n$ cents using at most four pennies. Then for some integer $n \geq 1$, $n+1$ should take no more than 4 pennies as well. Let $n \mod 5=k$ for some integer $k$. By definition of modulus we know that $k$must be smaller than the power of the modulus. Therefore $(n+1) \mod 5=k+1$ such that $k+1$ is some integer $ <5$ as well. Therefore by proof by induction, for all $n \geq 1$, you can make $n$ cents using at most four pennies

% ============================================
% ============================================
\collab{n/a} \nextprob{(4-5) Four Colors Suffice}
% ============================================
% ============================================

Read Chapters $4$ and $5$ of \emph{Four Colors Suffice}.

Use a proof by contradiction to prove that if an edge is removed from a
tree, then the resulting graph has two connected components.

EC:
Use a ``minimal criminal'' argument to prove this.

        \paragraph{Answer}

        Suppose not. That is suppose that if an edge is removed from a tree, then the resulting graph does not have two connected components. By definition of a tree graph, we know that for every vertex in a tree, there can only be
one edge conneting each vertex. Then if we were to remove one of those edges, there would be one disjoint vertex or sets of vertices from the tree. By definition of a graph, when there exists two or more seperate entities then they are considered connected componets. By the definition of disjoint, we know that if there are two or more disjoint items then they can be considered seperate entities. Therefore by removing $n$ edges from a tree, the result would be a graph of $n$ seperate entities and therefore be a graph of $n$ connected componets. This contradiciton shows that the supposition is false and therefore shows the proposition is true.

% %% ... the bibliography
% \newpage
% \bibliographystyle{acm}
% \bibliography{biblio}

\end{document}

