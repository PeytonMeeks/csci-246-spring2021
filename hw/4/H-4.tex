\documentclass{article}
\usepackage{../fasy-hw}

%% UPDATE these variables:
\renewcommand{\hwnum}{4}
\title{Discrete Structures, Homework \hwnum}
\author{Peyton Meeks}
\collab{n/a}
\date{due: 5 March 2021}

\begin{document}

\maketitle

This homework assignment should be
submitted as a single PDF file both to D2L and to Gradescope.

General homework expectations:
\begin{itemize}
    \item Homework should be typeset using LaTex.
    \item Answers should be in complete sentences and proofread.
    \item You will not plagiarize.
    \item List collaborators at the start of each question using the \texttt{collab} command.
    \item Put your answers where the \texttt{todo} command currently is (and
        remove the \texttt{todo}, but not the word \texttt{Answer}).
\end{itemize}


% ============================================
% ============================================
\collab{\todo{}} \nextprob{Good Proofs}
% ============================================
% ============================================

Look through proofs in this textbook, or other books / papers.  Define five
qualities that you think are common among good proofs. Provide citations to
examples.


\paragraph{Answer}

\todo{your answer here}



% ============================================
% ============================================
\collab{n/a} \nextprob{Max of a Subset}
% ============================================
% ============================================

Let $(B,\leq)$ be a totally ordered finite set. Prove the following
statement: For all nonempty subsets $A \subseteq B$, the following inequality
holds: $\max(A) \leq \max(B)$.

\paragraph{Answer}

Suppose there exists a subset $C \subseteq B$ for which $\max(C) > \max(B)$. If $B$ is truly a finite set then by definition of a finite set, $\max(B) \in B$. Since $C \subseteq B$, $\max(C)$ must exist within $B$. Therefore $\max(C)$ cannot be $> \max(B)$ and by proof by contradiction, for any subset $A \subseteq B$, $\max(A)$ must be $\leq \max(B)$.

% ============================================
% ============================================
\collab{\todo{}} \nextprob{Fibonacci}
% ============================================
% ============================================

The Fibonacci numbers are defined as follows:
$$
    F_i = \begin{cases}
		   0 & i=0\\
            1 & i \in \{1,2\} \\
            F_{i-1}+F_{i-2} & \text{otherwise}
          \end{cases}
$$

Prove $\sum_{i=1}^n F_i = F_{n+2}-1$.

\paragraph{Answer}

Let $n$ be an integer $\geq 1$ and assume that $\sum_{i=0}^n F_i = F_{n+2}-1$ is true. By the definiton of Fibonnaci numbers: $F_n =F_{n-1}+F_{n-2}$ for all integers $n>2$. Consider $F_{n+2}$, by the recursive formula $F{n+2}=F{n+1}+F_n$ and $F_{n+1} =F_n+F_{n-1}$. By using algerbra $F_{n+2}=F_n+F_{n-1}+F_n$. By inductive assumption, $\sum_{i=0}^n F_i = F_n+F_{n-1}+F_n-1$. Let $n=1$, such that $\sum_{i=0}^1 F_i = F_1+F_{1-1}+F_1-1$.Calulating the left side of the equation,  $\sum_{i=0}^n F_i =0+1=1$. Calculating the right side of the same equation,  $F_1+F_{1-1}+F_1-1= 1+0+1-1=1$. Both sides of the equation equal the same thus  $\sum_{i=0}^n F_i = F_{n+2}-1$ is true.

% ============================================
% ============================================
\collab{\todo{}} \nextprob{US Coins}
% ============================================
% ============================================

Consider the four smallest denominations of US coins: $D=\{1,5,10,25\}$.  Prove, using
induction, that, for each $n \geq 1$, you can make $n$ cents using at most four
pennies.

\paragraph{Answer}

\todo{your answer here}

% ============================================
% ============================================
\collab{\todo{}} \nextprob{Four Colors Suffice}
% ============================================
% ============================================

Read Chapters $4$ and $5$ of \emph{Four Colors Suffice}.

Use a proof by contradiction to prove that if an edge is removed from a
tree, then the resulting graph has two connected components.

EC:
Use a ``minimal criminal'' argument to prove this.

        \paragraph{Answer}

       Suppose there exists a tree in which removing an edge, the resulting graph does not have two connected points. By definition of a tree graph, we know that for every vertex in a tree, there can only be one edge conneting each vertex. Then if we were to remove one of those edges there would be one disjoint vertex from the tree. By definition of a graph when there exists two or more seperate entities then they are considered connected componets. Therefore by removing only one edge from a tree the result would be a graph of two seperate entities, or two connected components.

% ============================================
% ============================================
\collab{n/a}
\nextprob{Leonhard Euler}
% ============================================
% ============================================

Write a short (1-2 paragraph) biography of Leonhard Euler.
\textbf{In your own words}, describe who they are and why they are important in
the history of computer science.

If you use external resources, please provide
proper citations. If you do not use external sources, please write ``I did not
use any sources to write this biography'' as the last sentence of the
biography.

\paragraph{Answer}

Leonhard Euler was a engineer and mathematician in Switerland in the 1700's. He is known for many important discoveries in mathmatics which influenced the way that we understand math today. He devoted his life to the therory of mathmatics, with his total works adding up to 92 volumes. He is one of the most important people in the history of computer science due to these developments. Any algorithm used today more than likely uses one or more of Euler's therories. Some of his most noteable acomplishments are the development of the relation of a circle's diameter to its radius also known as $pi$, the use of $f(x)$ as the description of a function, and the use of $sigma ``\sum"$ to denote a summation.\footnote{“Leonhard Euler.” Wikipedia, Wikimedia Foundation, 6 Mar. 2021, \url{en.wikipedia.org/wiki/Leonhard_Euler}.} 

% %% ... the bibliography
% \newpage
% \bibliographystyle{acm}
% \bibliography{biblio}

\end{document}

