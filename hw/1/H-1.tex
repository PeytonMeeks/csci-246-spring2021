\documentclass{article}
\usepackage{../fasy-hw}
\usepackage{ wasysym }

%% UPDATE these variables:
\renewcommand{\hwnum}{1}
\title{Discrete Structures, Homework 1}
\author{Peyton Meeks (Peyton Meeks)}
\collab{n/a}
\date{due: 22 January 2021}

\begin{document}

\maketitle

This homework assignment should be
submitted as a single PDF file both to D2L and to Gradescope.

General homework expectations:
\begin{itemize}
    \item Homework should be typeset using LaTex.  (Note: if you are still
        having trouble with your setup, please reach out to the instructor and
        TA).
    \item Answers should be in complete sentences and proofread.
    \item You will not plagiarize.
    \item List collaborators at the start of each question using the
        \texttt{collab} command.
    \item Put your answers where the \texttt{todo} command currently is (and
        remove the \texttt{todo}, but not the word \texttt{Answer}).
\end{itemize}

% ============================================
% ============================================
\nextprob{Getting to Know Your Classmates}
\collab{n/a}
% ============================================
% ============================================

Find a different classmate for each of the following:
\begin{enumerate}
    \item Was born in the same month as you (year can be different).
        \paragraph{Answer} \todo{answer here}

    \item Has a shared hobby with you.
        \paragraph{Answer} \todo{answer here}

    \item Has the same middle initial as you.
        \paragraph{Answer} \todo{answer here}

    \item Lives in a different building than you.
        \paragraph{Answer} \todo{answer here}

    \item Has eaten at at least one restaurant or traveled to at least one city that you have not been
        (yet).
        \paragraph{Answer} \todo{answer here}

\end{enumerate}

% ============================================
% ============================================
\nextprob{Why Proofs?}
\collab{\todo{}}
% ============================================
% ============================================

Much of this class is spent learning how to prove things.  Explain why it is
important to you, as a computer scientist, to know how to prove things
mathematically.

\paragraph{Answer}

Being able to prove things is the key to humanities' evolvement of understanding. Knowing how something works so that it can be duplicated and studied in a controlled setting is how we have developed the modern tools that we have today. 


% ============================================
% ============================================
\nextprob{A Proof}
\collab{\todo{}}
% ============================================
% ============================================

Prove that $6\Z \subset 2\Z$.

\paragraph{Answer}

\todo{your answer here}


% ============================================
% ============================================
\nextprob{Grace Hopper}
\collab{\todo{}}
% ============================================
% ============================================

Write a short (1-2 paragraph) biography of Grace Hopper.
\textbf{In your own words}, describe who they are and why they are important in
the history of computer science.  If you use external resources, please provide
proper citations.

\paragraph{Answer}

\todo{your answer goes between these lines}

% ============================================
% ============================================
\nextprob{Bonus Question!}
\collab{\todo{}}
% ============================================
% ============================================

Use the `figure` environment to add a figure that provides the solution to
Exercises set 1.4, Problem 4.  Your figure can be hand drawn and scanned, or can
be made using a tool such as Inkscape.

\paragraph{Answer}

\todo{whenever you add a figure, be sure to add a reference to the figure too!
(otherwise, the reader might forget to look at the figure)}

\end{document}

